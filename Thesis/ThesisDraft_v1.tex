\documentclass[12pt, oneside]{book}
\usepackage{graphicx}  % this is for includegraphics

\usepackage{setspace}
\onehalfspacing % this sets spacing to 1.5 


\usepackage{amsmath}
\usepackage{amsthm} % for theorems, lemmas etc
\usepackage{amsfonts}
\usepackage{lipsum} % to generate the lipsum random text in the sample


\usepackage[colorlinks=true, urlcolor=blue, pdfborder={0 0 0}]{hyperref}
\hypersetup{
     colorlinks   = true,
     citecolor    = blue
}

\theoremstyle{plain}
\newtheorem{theorem}{Theorem}[section]
\newtheorem{proposition}[theorem]{Proposition}
\newtheorem{lemma}[theorem]{Lemma}
\newtheorem{corollary}[theorem]{Corollary}
\newtheorem{fact}[theorem]{Fact}

\theoremstyle{definition}
\newtheorem{definition}[theorem]{Definition}
\newtheorem{example}[theorem]{Example}
\newtheorem{remark}[theorem]{Remark}
\newtheorem{remarks}[theorem]{Remarks}

\newcommand{\Cov}{\mathrm{Cov}}
\newcommand{\Var}{\mathrm{Var}}

\begin{document}

\begin{titlepage}
\begin{center}
        \vspace{-2cm}
Mathematical Finance MSc Dissertation MTH775P, 2018/19 
		\\
        \Huge
        \textbf{Disquisitiones Arithmetic\ae}
        \\        
        \vspace{0.4cm}
        \LARGE
        High Performance Computing techniques for numerically solving financial PDEs
        \\        
        \vspace{0.4cm}        
        \textbf{Mustafa Berke Erdis, ID 180883925}% student name and number        
        \\
        \large Supervisor: Dr. Sebastian del Bano Rollin
        \\
        \vspace{0.9cm}
        \includegraphics[scale=0.3]{QMCrest.png}
        \\
        \vspace{0.9cm}        
        \LARGE 
        A thesis presented for the degree of\\
        Master in Sciences in \emph{Mathematical Finance}\\
        \vspace{0.7cm}        
        \Large
        School of Mathematical Sciences\\ 
        and \emph{School of Economics and Finance}\\
        Queen Mary University of London \\
    \end{center}
\end{titlepage}


\chapter*{Declaration of original work}
\begin{flushright}
This declaration is made on \today.
\end{flushright}


{\bf Student's Declaration:}
I, Mustafa Berke Erdis, hereby declare that the work in this thesis 
is my original work. I have not copied from any other students' work, work of 
mine submitted elsewhere,  or from any other sources except where due reference or acknowledgement is made explicitly in the text, nor has any part been written for me by another person.

Referenced text has been flagged by:
\begin{enumerate}
\item Using italic fonts, {\bf and} % LaTeX: {\it text}  
\item using quotation marks ``\ldots '', {\bf and}
\item explicitly mentioning the source in the text.
\end{enumerate}

%This excludes any definitions known from your modules or undethat can be found in an undergraduate text book.

\newpage

\thispagestyle{empty}
        \begin{flushright}
                This work is dedicated to my dog Charles Frederick.
        \end{flushright}
\vspace{\stretch{2}}\null



\chapter*{Acknowledgements}
Here you thank people that have helped you in the journey. \\
\lipsum[100] % replace this by your text

\chapter*{Abstract}
\begin{center}
\small 
Here you write a short summary, around 10 lines, of your work. \\
\lipsum[100]% replace this by your text
\end{center}       


\chapter*{Preface}
Here  you write a summary of the work. A paragraph on the motivation, previous work, then maybe a brief chapter by chapter summary. 

\lipsum[100]% replace this by your text



\begin{flushright}
Queen Mary University of London\\
12${}^{\text{th}}$ August 2019
\end{flushright}


\tableofcontents

\chapter{Introduction}

Partial differential equations o


Numerical partial differential equations is a large area of study. The subject includes components in the areas of applications, mathematics and computers. These three aspects of a problem are so strongly tied together that it is virtually impossible to consider the applied aspect of a problem without considering at least some of the mathematical counting aspects of that problem.[j.w. thomas]

The mathematical theory of partial differential equations describing financial markets plays an important role in mathematical finance. In most cases these equations are too complicated to be solved explicitly, therefore different methods of finding an approximate numerical solution is needed.

The most common framework is finite difference which tries to find approximate solutions to the problem at a discrete set of points, normally on a rectangular grid of points. It is simple to construct and analyse but can compromise performance because of increased computational complexity when there are high dimensions. The alternate direction implicit (ADI) method is used to numerically solve two dimensional parabolic PDEs. ADI schemes give us advantages of implicit finite difference method and computationally only requires to solve tridiagonal matrices \cite{thomas}. Finally, the Monte Carlo method is used to find the numerical solution when dimensions are too high by calculating an expectation (Feynman-Kac Theorem) \cite{klebaner}.

The existing numerical methods for partial differential equations are all constrained by the computational complexity. Motivated by present results and methods employed in high performance computing, we believe there are interesting and challenging topics in numerical solutions of PDEs for finance.

\section{Parabolic Partial Differential Equations}\label{Parabolic Partial Differential Equations}
\lipsum[5]
\subsection{Heat Equation}
\lipsum[5]
\subsection{Black-Scholes Equation}
\lipsum[5]
\section{Numerical Methods for Parabolic Differential Equations}
\lipsum[5]
\subsection{Explicit Method}
\lipsum[5]
\subsection{Implicit Method}
\lipsum[5]
\subsection{Theta Method and Crank-Nicholson Method}
\lipsum[5]
\subsection{Rannacher Trick}
\lipsum[5]
\subsection{Monte-Carlo Simulation}
\lipsum[5]

\section{Motivation for this work}
The idea of this project is to study how to take advantage of this parallelism and explore how much faster we can make these calculations.
\\
Being fast when evaluating new information is crucial for operations of hedge funds and investment banks. The aim of this project is to utilize High Performance Computing techniques to speed up the existing numerical methods using hardware and software that can be installed in a trading floor. Industry experience is the driver for the project to make an impact.

\begin{theorem}[{\cite[Theorem 2.3]{Petri}, see also \cite[pg. 45]{BlackScholes}}]\label{PetriTheorem}
The Gramm matrix for $E_8$ is:
$$
\begin{pmatrix}
2	&-1&0	&0	&0	&0	&0	&0\\
-1	&2	&-1	&0	&0	&0	&0	&0\\
0	&-1	&2	&-1	&0	&0	&0	&-1\\
0	&0	&-1	&2	&-1	&0	&0	&0\\
0	&0	&0	&-1	&2	&-1	&0	&0\\
0	&0	&0	&0	&-1	&2	&-1	&0\\
0	&0	&0	&0	&0	&-1	&2	&0\\
0	&0	&-1	&0	&0	&0	&0	&2
\end{pmatrix}.
$$
\end{theorem}


Recall the theorem of Petri \ref{PetriTheorem}
Look at section \ref{contentsofthesis}.

\subsection{The problem of numerical solutions}
Numerical analysis and computer simulations will be undertaken to put theory and observation together to gain insight into the workings of numerical solutions of partial differential equations.

We plan to develop the methods used for heat equation (${u_t(x,t)=u_{xx}(x,t)}$) as our basis point. As we go further into the project, the plan is to extend to Black-Scholes model ($\frac{\partial V}{\partial t} + \frac{1}{2}\sigma^2 S^2 \frac{\partial^2 V}{\partial S^2} = r(V - S \frac{\partial V}{\partial S})$)  and variations of Black-Scholes with increasing complexity such as the Multi-Asset Black-Scholes Model and Heston Stochastic Volatility Model.

First step is to write a simple version on a simple framework that can be calculated by hand and with Excel. Following the verifications, next step is porting the simple version in a high level programming language like python for prototyping and validating all calculations. The penultimate step is moving into a low level programming language such as C++ and utilize high performance computing principles.

High performance computing techniques that can be implemented for CPUs are pipelining and use of SSE/SIMD\cite{kusswurm} registers with Advanced Vector Extensions(AVX 512), multithreading with Open Multi-Processing(OpenMP) and compiler intrinsics. In the case of General Purpose GPUs, CUDA or Open Computing Language(OpenCL) can be utilized but can be challenging because of requirement of delicate memory management.

The project will be finalized by comparing the efficiency and speed of different implementations.


\subsection{The approach to optimizations}
\subsubsection{32 bit vs 64 bit}
\lipsum[5]
\subsubsection{Compilers VS/gcc/Intel}
\lipsum[5]
\subsubsection{AVX/Intrinsics}
\lipsum[5]
\subsubsection{Multithreading/OpenMP}
\lipsum[5]	
\subsubsection{CUDA}
\lipsum[5]	
\subsection{Timing the Code}
\subsubsection{Windows API}
\lipsum[5]
\subsubsection{Chrono Library}
\lipsum[5]

\chapter{Work Done Optimizing}

\section{Parallelizing Tridiagonal Systems}
\lipsum[5]



\chapter{Conclusions}
\lipsum[5]

\appendix
\chapter{Implementation of the {\tt BarrierOptionCVA} class}
\lipsum[100]
\chapter[shorter running title]{Additional details on the Gundermanian determinant}
\lipsum[100]



\bibliographystyle{plain}
\begin{thebibliography}{12}
\bibitem{thomas} Thomas, James W. {\it Numerical Partial Differential Equations: Finite Difference Methods.}, pp. 164, Springer, 1998.

\bibitem{klebaner}Klebaner, Fima C. {\it Introduction to Stochastic Calculus with Applications.}, pp. 155, Imperial College Press, 2005.

\bibitem{kusswurm}Kusswurm, Daniel. {\it Modern x86 Assembly Language Programming: 32-Bit, 64-Bit, SSE, and AVX.} Apress, 2015.
\bibitem[P99]{Petri}
   William Petri, 
  {\it Analysis of infinitely generated frog complexes},
	Rendicoti Ran\ae \ Analysorum, 234 {\bf (4)}, 34--21, 2015
\bibitem[Ross]{Ross}
  Sheldon Ross, {\it 
  \href{https://www-dawsonera-com.ezproxy.library.qmul.ac.uk/abstract/9781139069694}{An Elementary Introduction to Mathematical Finance}},
	3rd Edition, Cambridge University Press, 2011
	\bibitem[Hull]{Hull}
	John C. Hull, 
	{\it \href{https://www-dawsonera-com.ezproxy.library.qmul.ac.uk/abstract/9781447930419}{Options, Futures, and Other Derivatives}},
	8th Edition, Pearson Education, 2011
	\bibitem[BS]{BlackScholes}
	Fischer Black and  Myron Scholes,
	{\it \href{https://www.cs.princeton.edu/courses/archive/fall09/cos323/papers/black_scholes73.pdf}
	{The Pricing of Options and Corporate Liabilities}},
	Journal of Political Economy 81 {\bf 3}, 637--654,  (1973)
	
\end{thebibliography}



\end{document}